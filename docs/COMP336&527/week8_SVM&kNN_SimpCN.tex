\documentclass[11pt]{article}

% ------------------ 中文相关设置(pdfLaTeX + CJK)--------------------
\usepackage[utf8]{inputenc}  % 使得LaTeX可以识别UTF-8编码
\usepackage[T1]{fontenc}     % 用 T1 编码输出可容纳更多字符
\usepackage{CJKutf8}         % 启用CJKutf8宏包来支持中文

% ------------------ 其他常用宏包 --------------------
\usepackage{amsmath,amssymb,amsthm,amsfonts}
\usepackage{graphicx}
\usepackage{geometry}
\usepackage{hyperref}
\geometry{a4paper, margin=1in}
\linespread{1.2}

\title{\textbf{Support Vector Machines and k-Nearest Neighbors (Lecture Notes)}}
\author{根据 \\ 纽约大学柯朗数学研究所 Mehryar Mohri 的 \textit{Foundations of Machine Learning} \\ 利物浦大学Dom Richards的\textit{COMP336/COMP529 Big Data Analytics Lecture Note} \\ 编写 \\ ChatGPT-4o 翻译}
\date{}

\begin{document}
\begin{CJK}{UTF8}{gbsn}  % 开始CJK环境以支持中文

\maketitle

% -------------------------------------------------------------------
% 1. 支持向量机(SVM)
% -------------------------------------------------------------------
\section{支持向量机(SVM)}

% -------------------------------------------------------------------
\subsection{线性可分的严格定义}

给定一个训练数据集
\[
\mathcal{D} = \{(\mathbf{x}_i, y_i)\}_{i=1}^n, \quad \mathbf{x}_i \in \mathbb{R}^d, \quad y_i \in \{-1, +1\},
\]
我们称该数据集在输入空间 \(\mathbb{R}^d\) 中是\textit{线性可分}的,若存在一个超平面由 \(\mathbf{w} \in \mathbb{R}^d\) 与标量 \(b \in \mathbb{R}\) 确定,使得对所有 \(i \in \{1,\ldots,n\}\) 均有
\[
y_i\bigl(\mathbf{w}^\top \mathbf{x}_i + b\bigr) > 0.
\]
更严格地,还可以要求存在某个正数 \(\gamma>0\),对所有 \(i\) 都有
\[
y_i\bigl(\mathbf{w}^\top \mathbf{x}_i + b\bigr) \ge \gamma.
\]
若满足此条件,则称 \(\gamma\) 为\textit{几何间隔}(geometric margin)。在线性可分情况下,我们期望在所有能将数据正确分类的超平面中,找到几何间隔最大的那一个。

% -------------------------------------------------------------------
\subsection{硬间隔线性可分情形下的优化问题}

在线性可分的理想情形下,SVM 试图求解以下\textbf{硬间隔最大化}问题:
\[
\min_{\mathbf{w}, b} \quad \frac{1}{2}\|\mathbf{w}\|^2
\]
\[
\text{subject\ to}\quad y_i(\mathbf{w}^\top \mathbf{x}_i + b) \geq 1, \quad i=1,\ldots,n.
\]
将几何间隔定义为 \(\frac{2}{\|\mathbf{w}\|}\),最小化 \(\frac{1}{2}\|\mathbf{w}\|^2\) 等价于最大化几何间隔(倒数)。 

% -------------------------------------------------------------------
\subsubsection{拉格朗日乘子法与对偶问题}

为便于求解,我们将上述带约束的原问题转化为对偶问题。构造拉格朗日函数:
\[
\mathcal{L}(\mathbf{w}, b, \boldsymbol{\alpha}) 
= \frac{1}{2}\|\mathbf{w}\|^2 
- \sum_{i=1}^n \alpha_i \Bigl[y_i(\mathbf{w}^\top \mathbf{x}_i + b) - 1\Bigr],
\]
其中 \(\alpha_i \ge 0\) 为拉格朗日乘子(Lagrange multipliers)。接下来,对 \(\mathbf{w}\) 与 \(b\) 分别求偏导并令其为 0,可得
\[
\frac{\partial \mathcal{L}}{\partial \mathbf{w}} = \mathbf{w} - \sum_{i=1}^n \alpha_i y_i \mathbf{x}_i = 0
\quad \Longrightarrow \quad 
\mathbf{w} = \sum_{i=1}^n \alpha_i y_i \mathbf{x}_i,
\]
\[
\frac{\partial \mathcal{L}}{\partial b} = -\sum_{i=1}^n \alpha_i y_i = 0
\quad \Longrightarrow \quad 
\sum_{i=1}^n \alpha_i y_i = 0.
\]
将上述结果代入拉格朗日函数,可得到\textbf{对偶问题}:
\[
\max_{\boldsymbol{\alpha}} \quad \sum_{i=1}^n \alpha_i 
- \frac{1}{2}\sum_{i,j=1}^n \alpha_i \alpha_j y_i y_j \mathbf{x}_i^\top \mathbf{x}_j
\]
\[
\text{subject\ to} \quad \alpha_i \ge 0, \quad \sum_{i=1}^n \alpha_i y_i = 0.
\]
这是一个\textbf{二次规划}(Quadratic Programming, QP)问题,采用成熟的优化方法可高效求解。

% -------------------------------------------------------------------
\subsubsection{KKT 条件}
\label{subsec:kkt}

\textbf{KKT 条件}(Karush-Kuhn-Tucker conditions)是最优解 \((\mathbf{w}^*, b^*, \boldsymbol{\alpha}^*)\) 必须满足的一组必要条件,包括:
\begin{enumerate}
    \item \textbf{原始可行性(Primal feasibility)}:
    \[
    y_i \bigl(\mathbf{w}^\top \mathbf{x}_i + b\bigr) - 1 \;\;\geq\;\; 0,\quad \forall i;
    \]
    \item \textbf{对偶可行性(Dual feasibility)}:
    \[
    \alpha_i \;\;\geq\;\; 0,\quad \forall i;
    \]
    \item \textbf{互补松弛(Complementary slackness)}:
    \[
    \alpha_i\;\Bigl[\,y_i(\mathbf{w}^\top \mathbf{x}_i + b) - 1\,\Bigr] \;=\; 0,\quad \forall i;
    \]
    \item \textbf{站立性(Stationarity)}:
    \[
    \mathbf{w} - \sum_{i=1}^n \alpha_i y_i \mathbf{x}_i = 0,\quad
    \sum_{i=1}^n \alpha_i y_i = 0.
    \]
\end{enumerate}
根据互补松弛条件可知,若 \(\alpha_i>0\),则 \(y_i(\mathbf{w}^\top \mathbf{x}_i + b) - 1 = 0\),这些对应样本称为\textit{支持向量};若 \(\alpha_i=0\),则不一定严格落在边界上。

% -------------------------------------------------------------------
\subsection{软间隔与正则化}

在现实任务中,线性可分往往不成立,因此需允许少量分类错误。通过引入松弛变量 \(\xi_i \ge 0\) 并加入正则化参数 \(C>0\),得到\textbf{软间隔}优化问题:
\[
\min_{\mathbf{w}, b, \boldsymbol{\xi}} \quad \frac{1}{2}\|\mathbf{w}\|^2 + C \sum_{i=1}^n \xi_i
\]
\[
\text{subject\ to} \quad y_i(\mathbf{w}^\top \mathbf{x}_i + b) \ge 1 - \xi_i, \quad \xi_i \ge 0.
\]
其中 \(C\) 用于平衡间隔的最大化和误分类的惩罚,实际应用中通常通过交叉验证来选择合适的 \(C\)。

% -------------------------------------------------------------------
\subsection{核函数与核技巧(Kernel Trick)}
\label{sec:kernel}

\subsubsection{高维映射的动机}

对于线性不可分的数据,可以构造一个映射
\[
\phi: \mathbb{R}^d \;\to\; \mathcal{H},
\]
将输入空间 \(\mathbb{R}^d\) 中的每个向量 \(\mathbf{x}\) 映射到某个(甚至无限维的)希尔伯特空间 \(\mathcal{H}\)。在 \(\mathcal{H}\) 中,我们可能获得线性可分的优势。SVM 在该空间中寻找超平面
\[
\mathbf{w}^\top \phi(\mathbf{x}) + b = 0.
\]
然而,显式地计算 \(\phi(\mathbf{x})\) 可能会非常昂贵或不可行(尤其在高维或无限维情形)。

\subsubsection{核函数(Kernel Function)}

若存在核函数
\[
k(\mathbf{x}_i, \mathbf{x}_j) \;=\; \phi(\mathbf{x}_i)^\top \phi(\mathbf{x}_j),
\]
则 SVM 的\textbf{对偶目标函数}只需要在训练过程中计算形如 \(k(\mathbf{x}_i, \mathbf{x}_j)\) 的内积,而\textit{不需要}显式构造 \(\phi(\cdot)\)。这个过程被称为\textbf{核技巧}(kernel trick)。

\subsubsection{常见核函数}

\paragraph{线性核(Linear Kernel)}
\[
k(\mathbf{x}_i, \mathbf{x}_j) = \mathbf{x}_i^\top \mathbf{x}_j.
\]
这是最简单的核函数,对应的映射 \(\phi(\mathbf{x})\) 就是 \(\mathbf{x}\) 本身。在线性数据、或高维稀疏数据场景中常被使用。

\paragraph{多项式核(Polynomial Kernel)}
\[
k(\mathbf{x}_i, \mathbf{x}_j) 
= \bigl(\mathbf{x}_i^\top \mathbf{x}_j + c\bigr)^p,
\]
其中 \(c \ge 0\), \(p\) 为多项式次数。高阶多项式核可以捕捉更复杂的决策边界,维数易爆炸,核技巧大幅降低了这种复杂度。

\paragraph{高斯核(RBF Kernel)}
\[
k(\mathbf{x}_i, \mathbf{x}_j) 
= \exp\Bigl(-\frac{\|\mathbf{x}_i - \mathbf{x}_j\|^2}{2\sigma^2}\Bigr).
\]
对应了将数据映射到无穷维空间,RBF 核具有平滑和局部特性,是最常用的核之一。

\paragraph{Sigmoid 核(神经网络核)}
\[
k(\mathbf{x}_i, \mathbf{x}_j) 
= \tanh\bigl(\alpha \,\mathbf{x}_i^\top \mathbf{x}_j + c\bigr).
\]
与两层神经网络的激活函数类似,但并非总是满足 Mercer 条件,需要根据具体超参数判断是否正定。

一旦选择合适的核函数,SVM 的对偶问题变为
\[
\max_{\boldsymbol{\alpha}} \quad \sum_{i=1}^n \alpha_i 
- \frac{1}{2}\sum_{i,j=1}^n \alpha_i \alpha_j\,y_i y_j\, k(\mathbf{x}_i, \mathbf{x}_j),
\]
并得到决策函数
\[
f(\mathbf{x}) 
= \text{sgn}\Bigl(\sum_{i=1}^n \alpha_i y_i\,k(\mathbf{x}, \mathbf{x}_i) + b\Bigr).
\]

% -------------------------------------------------------------------
\subsection{再生核希尔伯特空间(RKHS)}
\label{subsec:rkhs}

\subsubsection{RKHS 的定义与再生性质}

考虑一个希尔伯特空间 \(\mathcal{H}\) 上定义的内积 \(\langle \cdot,\cdot\rangle_{\mathcal{H}}\)。若存在一个核函数 \(k: \mathcal{X}\times\mathcal{X}\to\mathbb{R}\) 满足:
\[
k(\mathbf{x}, \mathbf{y}) 
= \bigl\langle k(\mathbf{x}, \cdot), \; k(\mathbf{y}, \cdot)\bigr\rangle_{\mathcal{H}},
\]
且对任何 \(f \in \mathcal{H}\) 与 \(\mathbf{x} \in \mathcal{X}\) 都满足
\[
f(\mathbf{x}) 
= \bigl\langle f,\; k(\mathbf{x}, \cdot)\bigr\rangle_{\mathcal{H}},
\]
则称 \(\mathcal{H}\) 为\textbf{再生核希尔伯特空间}(Reproducing Kernel Hilbert Space, RKHS),而 \(k\) 称为\textbf{再生核}(reproducing kernel)。

这里:
\[
k(\mathbf{x}, \cdot)
\]
是一个从 \(\mathcal{X}\) 到 \(\mathbb{R}\) 的函数,位于 \(\mathcal{H}\) 当中,使得核函数正好是它们在 \(\mathcal{H}\) 中的内积。此性质称为\textbf{再生性质(reproducing property)}。

\subsubsection{将 RKHS 引入 SVM 的动机}

\begin{itemize}
    \item \textbf{统一的理论框架:} RKHS 提供了核方法的严谨数学基础,使得“用核函数代替内积”的做法有了系统化解释。
    \item \textbf{代表定理(Representer Theorem):} 在许多核学习方法(包括带有平方范数正则化的 SVM)中,最优解 \(f^* \in \mathcal{H}\) 可以写成
    \[
    f^*(\mathbf{x}) 
    = \sum_{i=1}^n \alpha_i\, k(\mathbf{x}_i, \mathbf{x}).
    \]
    该定理说明,虽然 \(\mathcal{H}\) 可能是无限维,但最优解实际仅在训练点对应的核函数上取线性组合。
    \item \textbf{可解释的几何:} 在 RKHS 中,\(\|f\|_{\mathcal{H}}\) 可以解释为函数的复杂度度量。最小化 \(\|f\|_{\mathcal{H}}^2\) + 损失函数相当于在函数复杂度与经验误差之间求折中,这与 SVM 最大间隔或软间隔的思想相吻合。
\end{itemize}

\subsubsection{符号与公式含义}

在以上描述中:
\begin{itemize}
    \item \(\mathcal{X}\) 是输入空间,如 \(\mathbb{R}^d\)。
    \item \(\mathcal{H}\) 是一个希尔伯特空间(可能是无限维),其中的向量可以视为函数。
    \item \(k(\cdot,\cdot)\) 是\textbf{再生核},满足对任意 \(\mathbf{x}, \mathbf{y}\in\mathcal{X}\),
    \[
    k(\mathbf{x}, \mathbf{y}) 
    = \langle\,k(\mathbf{x}, \cdot),\; k(\mathbf{y}, \cdot)\rangle_{\mathcal{H}}.
    \]
    \item \(\langle \cdot,\cdot\rangle_{\mathcal{H}}\) 表示 \(\mathcal{H}\) 上的内积。
    \item \(f(\mathbf{x})\) 对应在 \(\mathcal{H}\) 中的某个向量(函数)与 \(k(\mathbf{x},\cdot)\) 的内积:
    \[
    f(\mathbf{x}) 
    = \langle f, \; k(\mathbf{x}, \cdot)\rangle_{\mathcal{H}}.
    \]
\end{itemize}
在核方法中,这一框架解释了为什么我们可以用 \(\mathbf{x}_i\) 与 \(\mathbf{x}_j\) 的核函数值(而非显式映射后的内积)来进行学习和推断,同时也说明了为什么存在如 SVM 对偶解这样的“稀疏”形式。

% -------------------------------------------------------------------
% 2. k-近邻(kNN)
% -------------------------------------------------------------------
\section{k-近邻(kNN)}

\subsection{算法流程}

对于给定的数据集 
\[
\{(\mathbf{x}_i, y_i)\}_{i=1}^n, \quad y_i \in \{-1, +1\},
\]
k-近邻算法的\textbf{基本步骤}可总结为:
\begin{enumerate}
    \item \textbf{确定距离度量:} 常见如欧式距离 \(d(\mathbf{x}_i, \mathbf{x}_j) = \|\mathbf{x}_i - \mathbf{x}_j\|_2\)。具体任务下也可选择更合适的度量。
    \item \textbf{寻找 \(k\) 个邻居:} 对目标点 \(\mathbf{x}\),计算它与训练集所有点的距离,选出与其距离最小的 \(k\) 个样本。
    \item \textbf{投票或加权投票:} 在分类任务中,根据这 \(k\) 个邻居的标签 \(\{y_i\}\) 进行多数表决或加权表决,得到最终预测类别。
\end{enumerate}

\subsection{数学视角下的近邻}

kNN 是一种\textbf{非参数方法}:它并不对训练数据进行显式的参数化拟合,而是在做预测时依赖数据“局部”分布。可将 kNN 看作对后验概率 \(P(Y|\mathbf{x})\) 的一个邻域估计:
\[
\hat{P}(y \mid \mathbf{x}) 
\approx \frac{1}{k}\sum_{\mathbf{x}_i \in \mathcal{N}_k(\mathbf{x})} \mathbf{1}_{\{y_i = y\}},
\]
其中 \(\mathcal{N}_k(\mathbf{x})\) 表示 \(\mathbf{x}\) 的 \(k\) 个近邻,\(\mathbf{1}_{\{\cdot\}}\) 为指示函数。

\subsection{距离度量及加权策略}

在实践中,距离度量对 kNN 影响较大。除了欧式距离,还可采用曼哈顿距离、切比雪夫距离等。对于特征维度不同或量纲差异大的数据,可做归一化或选用更专业的距离。投票时常使用加权形式,例如距离的倒数:
\[
w_i = \frac{1}{d(\mathbf{x}, \mathbf{x}_i) + \varepsilon},
\]
使离得更近的邻居权重更大。

\subsection{\(k\) 的选择}

通过\textbf{交叉验证}等方式可选定合适的 \(k\)。当 \(k\) 太小,容易过拟合;当 \(k\) 太大,则欠拟合并丧失局部决策优势。



\end{CJK}
\end{document}
