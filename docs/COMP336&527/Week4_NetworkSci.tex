\documentclass{article}
\usepackage{amsmath} 
\usepackage{graphicx} 
\usepackage{amssymb} 
\usepackage{geometry}
\geometry{margin=1in}
\usepackage{graphicx}

\title{\textbf{Network Science}}
\author{Jingyuan Sun \\ ChatGPT \ Claude \\ Based on Prof. Laszlo Barabasi's lecture notes & Prof Vassil Alexandrov's taught}

\date{Nov, 2024}

\begin{document}

\maketitle

\section{Intro to Network Science}
Network Science is the study of complex systems composed of interconnected parts.

\subsection{What is Complex?}
\begin{enumerate}
    \item Composed of many interconnected parts: a complex highway system.
    \item Characterized by a very intricate arrangement of components: complex machinery.
    \item So complicated as to be hard to understand or resolve: a complex problem.
\end{enumerate}

\subsection{Characteristics of Network Science}

\begin{itemize}
    \item \textbf{Interdisciplinary}: Combines knowledge and methods from multiple fields.
    \item \textbf{Empirical and Data-Driven}: Relies on real-world data to analyze and validate results.
    \item \textbf{Quantitative and Mathematical}: Uses mathematical models to study and predict system behaviors.
    \item \textbf{Computational}: Leverages computational tools to simulate and solve complex problems.
\end{itemize}

\subsection{Applications and Societal Impact}

\begin{itemize}
    \item \textbf{Healthcare}: Designing drugs, predicting and tracking the spread of viruses.
    \item \textbf{Security and Cybersecurity}: Preventing terrorism and improving online security.
    \item \textbf{Economics and Management}: Supporting businesses like Amazon and Google with network-based strategies.
    \item \textbf{Social Networks}: Analyzing platforms such as Facebook, Twitter, and LinkedIn.
\end{itemize}


\section{Random Networks}
Random networks are a fundamental class of networks in network science, often used as a benchmark for studying complex systems.

\subsection{Definition and Degree Distribution}
A \textbf{random network} consists of $N$ nodes, where each pair of nodes is connected with a probability $p$. This model is often referred to as the Erd\H{o}s-R\'enyi (ER) model.

Mathematically, the degree distribution $p_k$ describes the probability that a randomly chosen node has degree $k$. For a random network, $p_k$ follows a \textbf{binomial distribution}:
\begin{equation}
    p_k = \binom{N-1}{k} p^k (1-p)^{N-1-k},
\end{equation}
where $k$ is the degree of the node, $N$ is the total number of nodes, and $p$ is the probability of a connection between two nodes. 
why N - 1: In Erd\H{o}s-R\'enyi (ER) model, every node's connection most be N - 1(exclude itself).

\subsection{Sparse Networks and Poisson Approximation}
For large $N$ and small $p$, where the network is \textbf{sparse} (i.e., $k \ll N$), the binomial distribution can be approximated by the \textbf{Poisson distribution}:
\begin{equation}
    p_k = e^{-\langle k \rangle} \frac{\langle k \rangle^k}{k!},
\end{equation}
where $\langle k \rangle = p(N-1)$ is the average degree of the network.

\textit{Explanation:} In a random network, most nodes have approximately the same degree, distributed around the mean $\langle k \rangle$. The degree distribution is symmetric and sharply peaked.


\subsection{Characteristics of Random Networks}
\begin{itemize}
    \item Random networks are highly homogeneous: most nodes have a similar degree.
    \item Degree distribution follows a binomial or Poisson distribution for sparse networks.
    \item They are not robust to random node failure: removing random nodes can quickly break the network apart.
\end{itemize}

\section{Scale-Free Networks}
In contrast to random networks, \textbf{scale-free networks} exhibit a degree distribution that follows a \textbf{power law}. These networks are prevalent in real-world systems such as the internet, biological networks, and social networks.

\subsection{Power Law Degree Distribution}
The degree distribution $p_k$ of a scale-free network is given by:
\begin{equation}
    p_k \sim k^{-\gamma},
\end{equation}
where $\gamma$ is the degree exponent, typically in the range $2 < \gamma < 3$.

\textit{Explanation:} In scale-free networks, a few nodes (called \textbf{hubs}) have very high degrees, while most nodes have relatively few connections. This is in stark contrast to the uniform degree distribution of random networks.

\subsection{Hubs and Robustness}
The presence of hubs in scale-free networks gives rise to unique properties:
\begin{itemize}
    \item \textbf{Robustness to Random Failures:} Removing random nodes does not significantly affect the network structure because hubs remain intact.
    \item \textbf{Vulnerability to Targeted Attacks:} Removing hubs can fragment the network.
\end{itemize}

\subsection{Comparison with Random Networks}
The key difference between scale-free and random networks lies in their degree distributions:
\begin{itemize}
    \item \textbf{Random Networks:} Degree distribution is binomial or Poisson.
    \item \textbf{Scale-Free Networks:} Degree distribution follows a power law.
\end{itemize}


\subsection{Example of the World Wide Web (WWW)}
The WWW is modeled as a scale-free network, where a few hubs (highly connected websites) dominate the network. The probability of observing a node with degree $k$ decreases as $k$ increases, following the power law distribution:
\begin{equation}
    p_k \sim k^{-\gamma}.
\end{equation}

\textbf{Conclusion:} The scale-free nature of real-world networks makes them more robust to random failures but more susceptible to targeted attacks. Hubs play a crucial role in maintaining network connectivity.


\section{Barabási–Albert Model}

\subsection{Growth and Preferential Attachment}
The Barabási–Albert (BA) model introduces two fundamental mechanisms to explain the emergence of scale-free networks: \textbf{Growth} and \textbf{Preferential Attachment}. These mechanisms result in networks with a degree distribution following a power law.

\paragraph{Growth}
Unlike the ER (Erdős–Rényi) model, where the number of nodes \( N \) is fixed (static), the BA model assumes that networks expand over time. New nodes are continuously added to the network.

\paragraph{Preferential Attachment}
New nodes prefer to connect to nodes that already have a high degree. Mathematically, the probability \( \Pi(k_i) \) that a new node connects to an existing node \( i \) with degree \( k_i \) is proportional to \( k_i \):
\begin{equation}
\Pi(k_i) = \frac{k_i}{\sum_j k_j}.
\end{equation}

\paragraph{Mechanism}
The BA model can be summarized as follows:
\begin{enumerate}
    \item \textbf{Start with an initial network} of \( m_0 \) nodes.
    \item \textbf{Growth}: At each time step, a new node is added to the network and connects to \( m \) existing nodes.
    \item \textbf{Preferential Attachment}: The probability of connecting to node \( i \) depends on its degree \( k_i \), as defined in Equation (1).
\end{enumerate}

\paragraph{Resulting Degree Distribution}
The combination of growth and preferential attachment leads to a degree distribution that follows a power law:
\begin{equation}
p_k \sim k^{-\gamma}, \quad \text{where } \gamma \approx 3.
\end{equation}
This means that a few nodes (hubs) will have a very high degree, while most nodes will have a low degree.

\begin{figure}[h]
    \centering
    \includegraphics[width=0.5\textwidth]{images/BA_Model.png} % Replace with actual image
    \caption{}
\end{figure}

\paragraph{Real-World Examples}
The BA model explains the emergence of scale-free networks in various real-world systems:
\begin{itemize}
    \item \textbf{World Wide Web (WWW)}: Nodes represent web pages, and edges represent hyperlinks.
    \item \textbf{Citation Network}: Nodes represent papers, and edges represent citations.
    \item \textbf{Actor Network}: Nodes represent actors, and edges represent shared movies.
\end{itemize}

\paragraph{Intuitive Explanation}
\begin{itemize}
    \item \textbf{Growth}: Networks naturally expand as new nodes are added.
    \item \textbf{Preferential Attachment}: New nodes prefer to link to well-connected nodes (hubs), which causes these nodes to become even more connected.
\end{itemize}

This mechanism mimics the ``rich get richer'' phenomenon observed in many real-world systems.




\subsection{Degree Dynamics}

In the Barabási–Albert (BA) model, the degree of a node evolves over time following a specific growth law. This results from the combined effects of \textbf{growth} and \textbf{preferential attachment}. Nodes that join the network earlier accumulate more links over time, leading to a degree distribution with the \textbf{first-mover advantage}.

\paragraph{Growth Law and Derivation}
For a node \( i \), its degree \( k_i(t) \) at time \( t \) evolves according to the following differential equation:
\begin{equation}
\frac{\partial k_i}{\partial t} \propto \Pi(k_i) = A \frac{k_i}{\sum_j k_j},
\end{equation}
where:
\begin{itemize}
    \item \( \Pi(k_i) \) is the probability that a new node connects to node \( i \), proportional to its current degree \( k_i \),
    \item \( \sum_j k_j \) is the total degree of all nodes in the network.
\end{itemize}

Since the total degree in the network at time \( t \) is given by:
\begin{equation}
\sum_j k_j = 2mt,
\end{equation}
the growth equation simplifies to:
\begin{equation}
\frac{\partial k_i}{\partial t} = \frac{k_i}{2t}.
\end{equation}

\paragraph{Solving the Differential Equation}
The equation \( \frac{\partial k_i}{\partial t} = \frac{k_i}{2t} \) is separable. Rearranging:
\begin{equation}
\frac{\partial k_i}{k_i} = \frac{\partial t}{2t}.
\end{equation}

Integrating both sides from \( t_i \) (the time node \( i \) joins the network) to \( t \), and assuming \( k_i(t_i) = m \) (initial degree when the node joins), we get:
\begin{equation}
\int_{k_i(t_i)=m}^{k_i(t)} \frac{1}{k_i} \, \partial k_i = \int_{t_i}^{t} \frac{1}{2t'} \, \partial t'.
\end{equation}

The solution is:
\begin{equation}
\ln \frac{k_i(t)}{m} = \frac{1}{2} \ln \frac{t}{t_i}.
\end{equation}

Exponentiating both sides:
\begin{equation}
\frac{k_i(t)}{m} = \left( \frac{t}{t_i} \right)^{\frac{1}{2}},
\end{equation}
which simplifies to:
\begin{equation}
k_i(t) = m \left( \frac{t}{t_i} \right)^{\frac{1}{2}}.
\end{equation}

\paragraph{Interpretation}
The degree \( k_i(t) \) of a node grows proportionally to \( t^{1/2} \). Nodes introduced earlier (\( t_i \) small) accumulate more links over time, leading to the \textbf{first-mover advantage}. This mechanism explains the emergence of hubs in scale-free networks.

\paragraph{Summary}
The key results from the degree dynamics in the BA model are:
\begin{itemize}
    \item The degree \( k_i(t) \) of a node increases over time following a power law: \( k_i(t) \sim t^{1/2} \).
    \item Nodes added earlier have a significant advantage, resulting in the formation of \textbf{hubs}.
    \item This growth behavior leads to a scale-free network with a degree distribution that follows a power law.
\end{itemize}
\subsection{Degree Distribution}

\noindent\textbf{Notation and Setup.}  
Consider a growing network constructed as follows: at each time step $t$, a new node is introduced and connected to $m$ existing nodes chosen with probability proportional to their degree. Let $N(t)=t$ represent the total number of nodes at time $t$. Define $P(k,t)$ as the probability (or fraction of nodes) that a randomly selected node at time $t$ has degree $k$. In the stationary limit $t \to \infty$, we denote $P(k)=\lim_{t\to\infty} P(k,t)$.

\begin{definition}
The \emph{degree distribution} of the network is given by the stationary probability $P(k)$ that a node chosen uniformly at random has degree $k$.
\end{definition}

\noindent\textbf{Rate Equations.}  
At each time step, a single new node is added to the network. The number of nodes at time $t$ is $N(t)=t$. The dynamics of $P(k,t)$ can be described by a set of discrete-time rate equations. For $k>m$, we have:
\begin{equation}
(N+1)P(k,t+1) = N P(k,t) + \frac{k-1}{2}P(k-1,t) - \frac{k}{2}P(k,t).
\label{eq:rate1}
\end{equation}
$N(t) = t$ is the number of nodes when time = t; $P(k, t)$ means the ratio of the number of nodes which have a k degree when time = t : total nodes. k/2 means preferred connection probability
For $k=m$, since each newly introduced node arrives with degree $m$, the rate equation takes a slightly different form:
\begin{equation}
(N+1)P(m,t+1) = N P(m,t) + 1 - \frac{m}{2} P(m,t).
\label{eq:rate2}
\end{equation}

\noindent\textbf{Stationary State.}  
In the stationary limit $t \to \infty$, set $P(k,t+1)=P(k,t)=P(k)$ and $N=t$. Substituting into \eqref{eq:rate1} and \eqref{eq:rate2}, we obtain time-independent equations. In particular, \eqref{eq:rate1} reduces to a recursive relation for $k>m$:
\begin{equation}
P(k) = \frac{k-1}{k+2} P(k-1).
\label{eq:recursive}
\end{equation}
From \eqref{eq:rate2} and the stationary condition for $k=m$, one can determine the normalization and boundary conditions necessary to solve \eqref{eq:recursive}.

\begin{theorem}[Degree distribution of the Barabási–Albert model]
In the Barabási–Albert (BA) model with parameter $m\geq 1$, the stationary degree distribution is:
\begin{equation}
P(k) = \frac{2m(m+1)}{k(k+1)(k+2)}.
\end{equation}
For large $k$, this asymptotically behaves as:
\begin{equation}
P(k) \sim k^{-3}.
\end{equation}
\end{theorem}

\begin{proof}
First, we use \eqref{eq:rate2} in the stationary limit to solve for $P(m)$:
\[
(N+1)P(m) = NP(m) + 1 - \frac{m}{2}P(m) \implies P(m)=\frac{2}{m+2}.
\]

Next, from \eqref{eq:recursive}, for $k>m$:
\[
P(k) = \frac{k-1}{k+2}P(k-1).
\]
Iterating this recursion starting from $k=m+1$, we have:
\[
P(m+1) = \frac{m}{m+3} P(m), \quad P(m+2)=\frac{m+1}{m+4}P(m+1), \quad \text{etc.}
\]
This telescopes into a product yielding:
\[
P(k) = \frac{2}{m+2} \cdot \frac{m(m+1)}{(m+2)(m+3)(m+4)\cdots(k+2)}.
\]

Careful simplification shows that the general form for all $k\geq m$ is:
\[
P(k)=\frac{2m(m+1)}{k(k+1)(k+2)},
\]
after reindexing and simplifying the product terms. As $k \to \infty$, we have $P(k)\sim 2m(m+1)/k^3$, confirming the power-law decay with exponent $3$.
\end{proof}

\begin{remark}
The exponent $\gamma = 3$ is independent of $m$, indicating a universal scaling behavior. Furthermore, the degree distribution $P(k)$ does not depend on time $t$ or the total system size $N$, which implies that the network reaches a stationary scale-free state. The proportionality constant of the power-law distribution scales as $m^2$, ensuring that variations in $m$ affect only the prefactor, but not the scaling exponent.
\end{remark}

\subsection{The Necessity of Growth and Preferential Attachment}

In the Barabási–Albert (BA) model, \textbf{growth} and \textbf{preferential attachment} are two essential mechanisms for generating scale-free networks. Each mechanism contributes uniquely, and their combination is critical to reproducing the power-law degree distribution observed in real-world networks.

\paragraph{1. Growth Alone}
- \textbf{Mechanism}: Growth refers to the continuous addition of new nodes to the network. At each step, new nodes connect to existing nodes with a fixed number of edges.
- \textbf{Limitation}: Without preferential attachment, new nodes would connect randomly, and the degree distribution would resemble a \textbf{Poisson distribution}, as seen in Erdős–Rényi random networks.
  - This implies that most nodes would have a similar degree, and no "hubs" (high-degree nodes) would emerge.
  - The network would lack the long-tail distribution characteristic of scale-free networks.

\paragraph{2. Preferential Attachment Alone}
- \textbf{Mechanism}: Preferential attachment ensures that new nodes are more likely to connect to existing nodes with higher degrees. This mechanism embodies the "rich-get-richer" principle.
- \textbf{Limitation}: Without growth, the network remains static, with no new nodes being added. While preferential attachment can increase the disparity in node degrees, it does not create a dynamic network or expand its size over time.
  - Real-world networks, such as social media or the World Wide Web, expand continuously with new participants or pages being added.

\paragraph{3. Combined Mechanisms in the BA Model}
The combination of growth and preferential attachment overcomes these limitations:
1. \textbf{Growth} provides the dynamic expansion observed in real-world networks. New nodes continuously join, increasing the network's size and complexity.
2. \textbf{Preferential attachment} introduces heterogeneity into the degree distribution, allowing the emergence of hubs (nodes with exceptionally high degrees).
3. Together, they generate a network with:
   - Dynamic growth and expanding scale.
   - A long-tail degree distribution that follows a power law.

\paragraph{4. Real-World Evidence Supporting Both Mechanisms}
- \textbf{Internet}: New websites are continuously created (growth) and tend to link to well-established, highly connected sites like Google or Wikipedia (preferential attachment).
- \textbf{Citation Networks}: New academic papers are published regularly (growth) and often cite highly cited, influential papers (preferential attachment).
- \textbf{Social Networks}: New users join platforms like Facebook or Twitter (growth) and are more likely to follow popular influencers or celebrities (preferential attachment).

\paragraph{5. Mathematical Complementarity}
From a mathematical perspective:
1. \textbf{Growth provides the initial conditions}: Growth introduces time-dependent terms into the degree equations (e.g., \( k_i(t) = m(t/t_i)^{1/2} \)).
2. \textbf{Preferential attachment creates heterogeneity}: The selection probability \( \Pi(k) \propto k \) ensures that high-degree nodes grow faster, resulting in a non-uniform, power-law degree distribution.

\subsection{Measuring Preferential Attachment}

In the Barabási–Albert (BA) model, preferential attachment plays a crucial role. To verify this mechanism, it is necessary to measure the attachment probability \( \Pi(k) \), which represents the probability of a node with degree \( k \) gaining new links over time.

\paragraph{1. Measuring \( \Pi(k) \)}
The rate of degree growth for a node \( i \) is proportional to \( \Pi(k_i) \):
\begin{equation}
\frac{\partial k_i}{\partial t} \propto \Pi(k_i),
\end{equation}
where \( \Pi(k) \sim \frac{\Delta k_i}{\Delta t} \) represents the rate at which nodes with degree \( k \) acquire new connections in a fixed time interval \( \Delta t \).

To measure \( \Pi(k) \):
\begin{itemize}
    \item Track the degree of nodes at time \( t \) and \( t + \Delta t \).
    \item Calculate the number of new links \( \Delta k \) for nodes with degree \( k \) during \( \Delta t \).
    \item Plot \( \Delta k / \Delta t \) as a function of \( k \).
\end{itemize}

\paragraph{2. Reducing Noise}
To reduce noise in the data, compute the cumulative sum of \( \Pi(k) \) over all degrees less than \( k \):
\begin{equation}
\kappa(k) = \sum_{K < k} \Pi(K).
\end{equation}
This approach yields:
\begin{itemize}
    \item \textbf{No preferential attachment:} \( \kappa(k) \sim k \) (linear growth).
    \item \textbf{Linear preferential attachment:} \( \kappa(k) \sim k^2 \) (quadratic growth).
\end{itemize}


\paragraph{3. Empirical Results}
The following networks have been analyzed to measure \( \kappa(k) \) and \( \Pi(k) \):
\begin{itemize}
    \item \textbf{Citation Network:} Connections based on paper citations.
    \item \textbf{Internet:} Nodes represent websites.
    \item \textbf{Neuroscience Collaboration:} Collaborations in neuroscience research.
    \item \textbf{Actor Collaboration:} Networks of co-acting in movies.
\end{itemize}

These networks demonstrate \( \kappa(k) \sim k^2 \), confirming the presence of linear preferential attachment.

\paragraph{4. General Form of \( \Pi(k) \)}
The attachment probability can be expressed generally as:
\begin{equation}
\Pi(k) \approx A + k^\alpha, \quad \alpha \leq 1.
\end{equation}
Linear preferential attachment corresponds to \( \alpha = 1 \), explaining the emergence of scale-free properties in real-world networks.


\subsection{Nonlinear Preferential Attachment}

In nonlinear preferential attachment, the probability of a new node connecting to an existing node with degree \( k \) is proportional to:
\begin{equation}
\Pi(k) \sim k^\alpha,
\end{equation}
where \( \alpha \) is a parameter controlling the dependency of the attachment probability on the degree.

\paragraph{1. Cases Based on \( \alpha \):}
\begin{itemize}
    \item \( \alpha = 0 \): Random attachment, where the degree distribution follows a simple exponential function.
    \item \( \alpha = 1 \): Linear preferential attachment (Barabási–Albert model), generating a scale-free network with degree exponent \( \gamma = 3 \).
    \item \( 0 < \alpha < 1 \): \textbf{Sublinear Preferential Attachment}. New nodes slightly favor higher-degree nodes, leading to a stretched exponential degree distribution:
    \[
    p_k \sim k^{-\alpha} \exp\left(-\frac{2\mu(\alpha)}{\langle k \rangle (1-\alpha)} k^{1-\alpha}\right),
    \]
    and the maximum degree grows logarithmically:
    \[
    k_{\text{max}} \sim (\ln t)^{1/(1-\alpha)}.
    \]
    \item \( \alpha > 1 \): \textbf{Superlinear Preferential Attachment}. New nodes strongly prefer connecting to high-degree nodes, accelerating the "rich-get-richer" process and resulting in a winner-takes-all effect. The maximum degree grows linearly:
    \[
    k_{\text{max}} \sim t.
    \]
\end{itemize}

\paragraph{2. Simulation Results}
Numerical simulations demonstrate the impact of \( \alpha \) on the growth of the maximum degree:
\begin{itemize}
    \item \( \alpha = 0.5 \): \( k_{\text{max}} \sim (\ln t)^2 \).
    \item \( \alpha = 1 \): \( k_{\text{max}} \sim t^{1/2} \).
    \item \( \alpha = 2.5 \): \( k_{\text{max}} \sim t \).
\end{itemize}



\end{document}

